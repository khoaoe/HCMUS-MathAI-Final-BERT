% =========================================================================
%% CHƯƠNG 1: GIỚI THIỆU CHUNG VỀ BERT
%% =========================================================================
\section{Giới Thiệu Chung về BERT}
\label{sec:gioi_thieu_chung_bert}

\subsection{Bối cảnh và Tầm quan trọng của Mô hình Biểu diễn Ngôn ngữ}
\label{ssec:boi_canh_bieu_dien_ngon_ngu}
Trong lĩnh vực Xử lý Ngôn ngữ Tự nhiên (NLP), việc xây dựng các biểu diễn ngôn ngữ (language representations) hiệu quả luôn là một mục tiêu trung tâm. Các biểu diễn này thường ở dạng vector, cố gắng nắm bắt các đặc tính ngữ nghĩa và cú pháp của từ, cụm từ hoặc câu, từ đó làm nền tảng cho nhiều tác vụ NLP phức tạp hơn. 

Trước khi BERT ra đời, các phương pháp biểu diễn ngôn ngữ phổ biến bao gồm các word embeddings tĩnh như Word2Vec \cite{mikolov2013distributed} và GloVe \cite{pennington2014glove}. Để hiểu rõ hạn chế của các phương pháp này, hãy xét một ví dụ cụ thể: từ "bank" trong câu "I went to the bank to deposit money" (ngân hàng) và "The river bank was flooded" (bờ sông) sẽ có cùng một vector biểu diễn, mặc dù ý nghĩa hoàn toàn khác nhau. Đây chính là hạn chế cố hữu của biểu diễn tĩnh (static) đó là không thể nắm bắt được ngữ cảnh.

Để khắc phục nhược điểm này, các mô hình biểu diễn ngôn ngữ theo ngữ cảnh (contextual language representation models) đã được phát triển. Các mô hình dựa trên kiến trúc tuần tự như Mạng Nơ-ron Hồi quy (RNN) và Bộ nhớ Dài-Ngắn Hạn (LSTM) bắt đầu cho thấy khả năng nắm bắt thông tin ngữ cảnh. Tuy nhiên, các mô hình này gặp phải vấn đề "information bottleneck" - thông tin phải được nén qua một vector trạng thái ẩn có kích thước cố định, dẫn đến mất mát thông tin khi xử lý chuỗi dài.

ELMo (Embeddings from Language Models) \cite{peters2018deep} đã cố gắng giải quyết vấn đề này bằng cách kết hợp biểu diễn từ cả hai chiều. Cụ thể, ELMo huấn luyện hai LSTM độc lập: một đọc văn bản từ trái sang phải, một đọc từ phải sang trái, sau đó concatenate các biểu diễn. Tuy nhiên, đây chỉ là kết hợp "nông" (shallow), hai LSTM không thực sự tương tác với nhau trong quá trình học.

Sự ra đời của kiến trúc Transformer \cite{vaswani2017attention} với cơ chế tự chú ý (self-attention) đã mang lại một cuộc cách mạng. Khác với RNN phải xử lý tuần tự từng từ, Transformer có thể xử lý toàn bộ chuỗi song song, cho phép mô hình hóa các phụ thuộc tầm xa một cách trực tiếp. OpenAI GPT \cite{radford2018improving} đã tận dụng Transformer nhưng vẫn giữ hướng huấn luyện từ trái sang phải, giới hạn khả năng hiểu ngữ cảnh toàn diện.

\texttt{BERT (Bidirectional Encoder Representations from Transformers)} \cite{devlin2018bert}, được Google giới thiệu năm 2018, đã giải quyết triệt để vấn đề này. BERT sử dụng một chiến lược đột phá: che ngẫu nhiên một số từ trong câu và yêu cầu mô hình dự đoán chúng dựa trên ngữ cảnh hai chiều. Điều này cho phép BERT học được biểu diễn sâu sắc hai chiều (deeply bidirectional), mỗi từ được hiểu dựa trên toàn bộ ngữ cảnh xung quanh, không chỉ một phía.

\subsection{Mục tiêu và Đóng góp chính của BERT}
\label{ssec:muc_tieu_dong_gop_bert}
Mục tiêu chính của bài báo BERT \cite{devlin2018bert} là giới thiệu một phương pháp tiền huấn luyện (pre-training) các biểu diễn ngôn ngữ hai chiều, có khả năng nắm bắt thông tin ngữ nghĩa và cú pháp phong phú từ dữ liệu văn bản không gán nhãn. Đây là một bước đột phá quan trọng vì dữ liệu không gán nhãn rất dồi dào và dễ thu thập, trong khi dữ liệu có nhãn cho các tác vụ cụ thể thường khan hiếm và tốn kém hơn.

Những đóng góp chính của BERT bao gồm:
\begin{itemize}
    \item \textbf{Kiến trúc hai chiều sâu sắc (Deeply Bidirectional):} Đây là điểm khác biệt quan trọng nhất của BERT. Trong khi các mô hình trước đó như GPT chỉ nhìn ngữ cảnh một chiều, hoặc như ELMo kết hợp hai chiều một cách độc lập, BERT cho phép mỗi lớp của mô hình xem xét ngữ cảnh hai chiều đồng thời. Điều này có nghĩa là khi xử lý từ "bank" trong câu, BERT có thể đồng thời xem xét cả "went to the" (bên trái) và "to deposit money" (bên phải) để hiểu đây là "ngân hàng" chứ không phải "bờ sông".
    
    \item \textbf{Hai tác vụ tiền huấn luyện sáng tạo:}
    \begin{itemize}
        \item \textbf{Masked Language Model (MLM):} BERT che ngẫu nhiên 15\% các từ trong câu và yêu cầu mô hình dự đoán chúng. Ví dụ, từ câu "The cat sat on the mat", BERT có thể che từ "sat" thành "[MASK]" và học cách dự đoán từ đó dựa trên ngữ cảnh. Điều quan trọng là BERT sử dụng chiến lược 80-10-10: 80\% thời gian thay bằng [MASK], 10\% thay bằng từ ngẫu nhiên, 10\% giữ nguyên. Chiến lược này giúp giảm sự khác biệt giữa giai đoạn huấn luyện (có [MASK]) và sử dụng thực tế (không có [MASK]).
        
        \item \textbf{Next Sentence Prediction (NSP):} BERT học dự đoán xem hai câu có liên tiếp trong văn bản gốc hay không. Ví dụ, "The weather is nice today. Let's go for a walk" (IsNext) so với "The weather is nice today. Penguins live in Antarctica" (NotNext). Tác vụ này giúp BERT hiểu mối quan hệ giữa các câu, rất quan trọng cho các ứng dụng như hỏi đáp hay suy luận ngôn ngữ.
    \end{itemize}
    
    \item \textbf{Hiệu suất vượt trội:} BERT đã thiết lập kỷ lục mới trên 11 tác vụ NLP khác nhau. Trên GLUE benchmark, BERT\textsubscript{LARGE} đạt điểm trung bình 80.5, vượt xa mô hình tốt nhất trước đó (OpenAI GPT với 72.8). Trên SQuAD v1.1, BERT thậm chí vượt qua hiệu suất của con người (F1: 93.2 so với 91.2).
    
    \item \textbf{Tính linh hoạt và dễ áp dụng:} BERT được thiết kế với triết lý "pre-train once, fine-tune for everything". Sau khi tiền huấn luyện trên dữ liệu lớn (BookCorpus và Wikipedia), BERT có thể được tinh chỉnh cho bất kỳ tác vụ NLP nào chỉ bằng cách thêm một lớp đầu ra đơn giản. Quá trình fine-tune thường chỉ mất vài giờ trên GPU, so với hàng tuần cho việc huấn luyện từ đầu.
\end{itemize}

Sự thành công của BERT không chỉ nằm ở hiệu suất vượt trội mà còn ở việc nó mở ra một hướng nghiên cứu mới: sử dụng mô hình ngôn ngữ lớn được tiền huấn luyện làm nền tảng cho mọi tác vụ NLP. Điều này đã dẫn đến sự phát triển của cả một thế hệ mô hình mới như RoBERTa, ALBERT, và xa hơn là GPT-3.

%% =========================================================================
%% CHƯƠNG 2: KIẾN TRÚC BERT VÀ CÁC THÀNH PHẦN CỐT LÕI
%% =========================================================================
\section{Kiến trúc BERT và các Thành phần Cốt lõi}
\label{sec:kien_truc_bert}
Kiến trúc của BERT dựa trên kiến trúc Transformer \cite{vaswani2017attention}, cụ thể là phần bộ mã hóa (encoder). Sau đây nhóm sẽ phân tích từng thành phần cốt lõi và cách chúng kết hợp để tạo nên sức mạnh của mô hình.

\subsection{Tổng quan về Kiến trúc Transformer}
\label{ssec:tong_quan_transformer}
Transformer là một kiến trúc mạng nơ-ron được giới thiệu với mục tiêu ban đầu là cải thiện các mô hình dịch. Điểm đột phá của Transformer là loại bỏ hoàn toàn các cấu trúc tuần tự (RNN, LSTM) và thay thế bằng cơ chế attention, cho phép xử lý song song toàn bộ chuỗi.

Kiến trúc Transformer gồm hai phần chính:
\begin{itemize}
    \item \textbf{Encoder (Bộ mã hóa):} Xử lý chuỗi đầu vào và tạo ra biểu diễn ngữ cảnh
    \item \textbf{Decoder (Bộ giải mã):} Sinh ra chuỗi đầu ra dựa trên biểu diễn từ encoder
\end{itemize}

BERT chỉ sử dụng phần Encoder vì mục tiêu là tạo biểu diễn ngôn ngữ, không phải sinh văn bản. Mỗi khối encoder trong BERT bao gồm:
\begin{itemize}
    \item \textbf{Lớp Multi-Head Self-Attention:} Cho phép mô hình "chú ý" đến các vị trí khác nhau trong chuỗi khi xử lý mỗi từ. Ví dụ, khi xử lý từ "it" trong câu "The animal didn't cross the street because it was too tired", lớp attention giúp mô hình liên kết "it" với "animal" chứ không phải "street".
    
    \item \textbf{Lớp Feed-Forward Network:} Xử lý độc lập từng vị trí với cùng một mạng nơ-ron. Điều này giúp biến đổi phi tuyến các biểu diễn sau khi đã tổng hợp thông tin ngữ cảnh từ attention.
\end{itemize}

Mỗi lớp con đều có kết nối dư (residual connection) và chuẩn hóa lớp (layer normalization), giúp huấn luyện ổn định các mạng nơ-ron sâu (deep neural networks).

\begin{figure}[H]
    \centering
    \includegraphics[width=0.5\textwidth]{img/transformer_arch.png}
    \caption{Kiến trúc tổng quan của mô hình Transformer. BERT chỉ sử dụng phần Encoder (bên trái). Mỗi khối encoder chứa lớp Multi-Head Attention và Feed-Forward, được kết nối bằng residual connections và layer normalization.}
    \label{fig:transformer_architecture}
\end{figure}

\subsection{Cơ chế Tự chú ý (Self-Attention) - Trái tim của BERT}
\label{ssec:self_attention}
Self-attention là cơ chế cho phép mỗi từ trong chuỗi "nhìn" và tương tác với mọi từ khác để xây dựng biểu diễn của mình. Điều này khác biệt hoàn toàn với RNN chỉ có thể xem thông tin một cách tuần tự.

\subsubsection{Trực quan về Attention}
Hãy tưởng tượng đang đọc câu và cần hiểu từ "bank". Não bộ sẽ tự động "chú ý" đến các từ xung quanh như "river", "money", "deposit" để xác định nghĩa. Self-attention mô phỏng chính quá trình này.

\subsubsection{Công thức toán học}
Với đầu vào là các vector biểu diễn của từ, self-attention tính toán ba ma trận:
\begin{itemize}
    \item \textbf{Query (Q):} "Tôi đang tìm kiếm thông tin gì?"
    \item \textbf{Key (K):} "Tôi có thông tin gì để cung cấp?"
    \item \textbf{Value (V):} "Thông tin thực sự tôi sẽ truyền đi là gì?"
\end{itemize}

Công thức Scaled Dot-product Attention:
$$ \text{Attention}(Q, K, V) = \text{softmax}\left(\frac{QK^T}{\sqrt{d_k}}\right)V $$

Trong đó:
\begin{itemize}
    \item $QK^T$: Tính điểm tương đồng giữa queries và keys (ma trận attention scores)
    \item $\sqrt{d_k}$: Hệ số chuẩn hóa để tránh gradient vanishing khi $d_k$ lớn
    \item $\text{softmax}$: Chuyển điểm số thành xác suất (tổng = 1)
    \item Nhân với $V$: Tổng hợp thông tin dựa trên trọng số attention
\end{itemize}

\subsubsection{Ví dụ minh họa}
Xét câu: "The cat sat on the mat"
Khi tính attention cho từ "sat":
\begin{enumerate}
    \item Query của "sat" được so sánh với Key của mọi từ
    \item Điểm số cao với "cat" (chủ ngữ) và "mat" (địa điểm)
    \item Sau softmax, ta có trọng số attention (ví dụ: cat=0.3, sat=0.1, on=0.1, the=0.15, mat=0.35)
    \item Biểu diễn mới của "sat" = tổng có trọng số của các Value
\end{enumerate}

\begin{figure}[H]
    \centering
    \includegraphics[width=0.5\textwidth]{img/scaled_dotproduct.png}
    \caption{Cơ chế Scaled Dot-Product Attention. MatMul thực hiện phép nhân ma trận $QK^T$, Scale chia cho $\sqrt{d_k}$, Mask (tùy chọn) che các vị trí không hợp lệ, Softmax chuẩn hóa thành xác suất, và MatMul cuối cùng tính tổng có trọng số của Values.}
    \label{fig:scaled_dot_product_attention}
\end{figure}

\subsection{Chú ý Đa đầu (Multi-Head Attention) - Sức mạnh từ sự đa dạng}
\label{ssec:multi_head_attention}
Một cơ chế attention duy nhất chỉ có thể nắm bắt một loại mối quan hệ. Multi-Head Attention giải quyết hạn chế này bằng cách chạy nhiều attention song song, mỗi cái tập trung vào khía cạnh khác nhau.

\subsubsection{Ý tưởng chính}
Thay vì có một attention lớn, ta chia thành $h$ attention nhỏ hơn (heads). Mỗi head học một "góc nhìn" khác:
\begin{itemize}
    \item Head 1 có thể học quan hệ cú pháp (chủ ngữ - vị ngữ)
    \item Head 2 có thể học quan hệ ngữ nghĩa (từ đồng nghĩa)
    \item Head 3 có thể học khoảng cách (từ gần nhau)
    \item ...
\end{itemize}

\subsubsection{Công thức toán học}
$$ \text{MultiHead}(Q, K, V) = \text{Concat}(\text{head}_1, ..., \text{head}_h)W^O $$
$$ \text{với } \text{head}_i = \text{Attention}(QW_i^Q, KW_i^K, VW_i^V) $$

Trong đó:
\begin{itemize}
    \item $W_i^Q, W_i^K, W_i^V \in \mathbb{R}^{d_{model} \times d_k}$: Ma trận chiếu cho head thứ $i$
    \item $W^O \in \mathbb{R}^{hd_v \times d_{model}}$: Ma trận chiếu đầu ra
    \item $d_k = d_v = d_{model}/h$: Kích thước mỗi head
\end{itemize}

BERT\textsubscript{BASE} sử dụng $h=12$ heads với $d_{model}=768$, nên mỗi head có kích thước $d_k=64$.

\subsubsection{Ví dụ thực tế}
Trong câu "The bank will not approve the loan", các head khác nhau có thể học:
\begin{itemize}
    \item Head 1: Liên kết "bank" với "loan" (quan hệ ngữ nghĩa tài chính)
    \item Head 2: Liên kết "will not" với "approve" (phủ định)
    \item Head 3: Liên kết "the" với danh từ theo sau (quan hệ ngữ pháp)
\end{itemize}

\begin{figure}[H]
    \centering
    \includegraphics[width=0.5\textwidth]{img/multihead.png}
    \caption{Cơ chế Multi-Head Attention. Đầu vào được chiếu thành $h$ bộ Q, K, V độc lập. Mỗi head thực hiện attention riêng, sau đó các kết quả được nối lại và chiếu qua $W^O$ để tạo đầu ra cuối cùng.}
    \label{fig:multi_head_attention}
\end{figure}

\subsection{Position-wise Feed-Forward Networks - Biến đổi phi tuyến}
\label{ssec:feed_forward_networks}
Sau khi tổng hợp thông tin qua attention, mỗi vị trí được xử lý qua một mạng FFN giống nhau:

$$ \text{FFN}(x) = \max(0, xW_1 + b_1)W_2 + b_2 $$

Đặc điểm quan trọng:
\begin{itemize}
    \item Cùng một FFN được áp dụng cho mọi vị trí (parameter sharing)
    \item Lớp ẩn có kích thước lớn hơn ($d_{ff}=3072$ so với $d_{model}=768$)
    \item hàm kích hoạt ReLU tạo tính phi tuyến
\end{itemize}

FFN đóng vai trò như bộ xử lý đặc trưng cục bộ sau khi đã có ngữ cảnh toàn cục từ attention.

\subsection{Residual Connections và Layer Normalization}
\label{ssec:residual_layer_norm}
Mỗi sublayer (attention hoặc FFN) được bọc bởi:
$$ \text{LayerNorm}(x + \text{Sublayer}(x)) $$

Lợi ích:
\begin{itemize}
    \item \textbf{Residual connection ($x +$):} Cho phép gradient flow trực tiếp, giúp huấn luyện mạng nơ-ron sâu
    \item \textbf{Layer normalization:} Chuẩn hóa theo chiều features, ổn định quá trình học
\end{itemize}

\subsection{Biểu diễn Đầu vào của BERT - Kết hợp thông tin đa chiều}
\label{ssec:input_representation_bert}
BERT xây dựng biểu diễn đầu vào bằng cách cộng ba loại embedding:

$$ \text{Input}_i = \text{TokenEmb}_i + \text{SegmentEmb}_i + \text{PositionEmb}_i $$

\subsubsection{Token Embeddings}
\begin{itemize}
    \item Sử dụng WordPiece tokenization với vocabulary $\approx$ 30,000 tokens
    \item Ví dụ: "playing" → ["play", "\#\#ing"]
    \item Giúp xử lý từ hiếm và từ mới (out-of-vocabulary)
\end{itemize}

\subsubsection{Segment Embeddings}  
\begin{itemize}
    \item Phân biệt câu A và câu B trong các tác vụ cặp câu
    \item Chỉ có 2 giá trị: $E_A$ cho câu đầu, $E_B$ cho câu thứ hai
    \item Ví dụ: "[CLS] How are you ? [SEP] I am fine [SEP]"
    \begin{itemize}
        \item Tokens của "How are you ?" nhận $E_A$
        \item Tokens của "I am fine" nhận $E_B$
    \end{itemize}
\end{itemize}

\subsubsection{Position Embeddings}
\begin{itemize}
    \item Transformer không có cấu trúc tuần tự nên cần thông tin vị trí
    \item BERT học position embeddings (khác với Transformer gốc dùng sinusoidal)
    \item Giới hạn 512 positions (có thể mở rộng khi cần)
\end{itemize}

\subsubsection{Special Tokens}
\begin{itemize}
    \item \textbf{[CLS]:} Token đầu tiên, biểu diễn cuối cùng dùng cho classification
    \item \textbf{[SEP]:} Phân tách câu và đánh dấu kết thúc
    \item \textbf{[MASK]:} Dùng trong pre-training MLM
    \item \textbf{[PAD]:} Padding để các chuỗi có cùng độ dài
\end{itemize}

\begin{figure}[H]
    \centering
    \includegraphics[width=0.9\textwidth]{img/input_representation.png}
    \caption{Biểu diễn đầu vào của BERT. Ví dụ cho cặp câu "He likes playing" và "My dog is cute". Ba loại embeddings được cộng theo từng vị trí để tạo input embedding cuối cùng. Token [CLS] ở đầu sẽ được dùng cho các tác vụ phân loại.}
    \label{fig:bert_input_representation}
\end{figure}

\subsection{Các phiên bản BERT}
\label{ssec:bert_versions}
Google phát hành hai phiên bản chính:

\begin{table}[H]
\centering
\caption{So sánh các phiên bản BERT}
\begin{tabular}{lccccc}
\toprule
\textbf{Phiên bản} & \textbf{Layers (L)} & \textbf{Hidden (H)} & \textbf{Heads (A)} & \textbf{Parameters} & \textbf{Training Time} \\
\midrule
BERT\textsubscript{BASE} & 12 & 768 & 12 & 110M & 4 days on 16 TPUs \\
BERT\textsubscript{LARGE} & 24 & 1024 & 16 & 340M & 4 days on 64 TPUs \\
\bottomrule
\end{tabular}
\end{table}

Sự khác biệt không chỉ về kích thước mà còn về khả năng học:
\begin{itemize}
    \item BERT\textsubscript{LARGE} có capacity lớn hơn, phù hợp cho tác vụ phức tạp
    \item BERT\textsubscript{BASE} hiệu quả hơn về tài nguyên, phù hợp cho deployment
    \item Cả hai đều vượt trội so với các mô hình trước đó
\end{itemize}


%% =========================================================================
%% CHƯƠNG 3: CÁC TÁC VỤ TIỀN HUẤN LUYỆN CỦA BERT
%% =========================================================================
\section{Các Tác vụ Tiền huấn luyện của BERT}
\label{sec:pre_training_tasks}
BERT được tiền huấn luyện đồng thời trên hai tác vụ tự giám sát (self-supervised): Masked Language Model (MLM) và Next Sentence Prediction (NSP). Các tác vụ này cho phép BERT học từ dữ liệu văn bản không cần gán nhãn thủ công.

\subsection{Masked Language Model (MLM) - Đột phá cho học hai chiều}
\label{ssec:mlm}

\subsubsection{Vấn đề cốt lõi}
Các mô hình ngôn ngữ truyền thống chỉ có thể học một chiều vì lý do đơn giản: nếu cho phép mô hình "nhìn" cả hai chiều khi dự đoán từ tiếp theo, nó sẽ "gian lận" bằng cách nhìn trực tiếp vào từ cần dự đoán. BERT giải quyết vấn đề này bằng cách: che ngẫu nhiên một số từ và yêu cầu mô hình dự đoán chúng từ ngữ cảnh.

\subsubsection{Chiến lược Masking 80-10-10}
BERT che 15\% tokens trong mỗi chuỗi, nhưng không phải lúc nào cũng thay bằng [MASK]:

\begin{itemize}
    \item \textbf{80\% trường hợp → [MASK]:} 
    \begin{itemize}
        \item Input: "The cat sat on the mat"
        \item Masked: "The cat [MASK] on the mat"
        \item Target: Dự đoán "sat"
    \end{itemize}
    
    \item \textbf{10\% trường hợp → Random token:}
    \begin{itemize}
        \item Input: "The cat sat on the mat"
        \item Masked: "The cat apple on the mat"
        \item Target: Dự đoán "sat" (không phải "apple")
    \end{itemize}
    
    \item \textbf{10\% trường hợp → Giữ nguyên:}
    \begin{itemize}
        \item Input: "The cat sat on the mat"
        \item Masked: "The cat sat on the mat"
        \item Target: Vẫn dự đoán "sat"
    \end{itemize}
\end{itemize}

\subsubsection{Tại sao cần chiến lược này?}
\begin{enumerate}
    \item \textbf{Vấn đề [MASK] token:} Trong quá trình fine-tuning và sử dụng thực tế, không có token [MASK]. Nếu chỉ train với [MASK], mô hình sẽ học cách xử lý [MASK] tốt nhưng kém với các từ thực.
    
    \item \textbf{Random replacement (10\%):} Buộc mô hình phải học biểu diễn tốt cho MỌI token, không chỉ [MASK]. Mô hình phải "nghi ngờ" mọi từ có thể bị thay thế.
    
    \item \textbf{Keep unchanged (10\%):} Đảm bảo mô hình cũng học cách biểu diễn các từ không bị thay đổi. Tạo cân bằng trong học tập.
\end{enumerate}

\subsubsection{Ví dụ huấn luyện MLM}
Câu gốc: "BERT revolutionized natural language processing in 2018"
\begin{enumerate}
    \item Chọn ngẫu nhiên 15\% tokens (giả sử chọn "revolutionized" và "2018")

    \item Áp dụng chiến lược:
    \begin{itemize}
        \item "revolutionized" → [MASK] (80\% chance)
        \item "2018" → "cat" (10\% chance - random)
    \end{itemize}
\end{enumerate}

=> Input cho BERT: "BERT [MASK] natural language processing in cat"

Mục tiêu: 
\begin{itemize}
    \item Tại vị trí [MASK]: dự đoán "revolutionized"
    \item Tại vị trí "cat": dự đoán "2018"
    \item Các vị trí khác: không tính loss
\end{itemize}

\subsubsection{Hàm loss cho MLM}
Loss chỉ được tính cho các vị trí bị mask:
$$ \mathcal{L}_{MLM} = -\sum_{i \in \text{masked}} \log P(w_i | \text{context}) $$

Trong đó $P(w_i | \text{context})$ là xác suất dự đoán đúng từ gốc tại vị trí $i$.

\subsection{Next Sentence Prediction (NSP) - Học quan hệ giữa câu}
\label{ssec:nsp}

\subsubsection{Động lực}
Nhiều tác vụ NLP quan trọng yêu cầu hiểu mối quan hệ giữa các câu:
\begin{itemize}
    \item \textbf{Question Answering:} Câu hỏi và đoạn văn có liên quan?
    \item \textbf{Natural Language Inference:} Câu A kéo theo câu B?
    \item \textbf{Paraphrasing:} Hai câu có cùng ý nghĩa?
\end{itemize}

MLM chỉ học quan hệ trong câu, NSP bổ sung khả năng học quan hệ giữa các câu.

\subsubsection{Cách thức hoạt động}
\begin{enumerate}
    \item \textbf{Tạo dữ liệu huấn luyện:}
    \begin{itemize}
        \item 50\% mẫu: Câu B thực sự theo sau câu A trong văn bản (label: IsNext)
        \item 50\% mẫu: Câu B là câu ngẫu nhiên từ corpus (label: NotNext)
    \end{itemize}
    
    \item \textbf{Input format:}
    \begin{verbatim}
    [CLS] Sentence A [SEP] Sentence B [SEP]
    \end{verbatim}
    
    \item \textbf{Prediction:} Sử dụng biểu diễn cuối cùng của [CLS] token qua một classifier:
    $$ P(\text{IsNext}) = \text{softmax}(W_{NSP} \cdot h_{[CLS]} + b_{NSP}) $$
\end{enumerate}

\subsubsection{Ví dụ cụ thể}
\textbf{Positive example (IsNext):}
\begin{itemize}
    \item A: "The weather forecast predicts heavy rain tomorrow."
    \item B: "I should bring an umbrella to work."
    \item Logic: B là hệ quả logic của A
\end{itemize}

\textbf{Negative example (NotNext):}
\begin{itemize}
    \item A: "The weather forecast predicts heavy rain tomorrow."
    \item B: "Python is a popular programming language."
    \item Logic: Không có quan hệ ngữ nghĩa
\end{itemize}

\subsubsection{Vai trò của [CLS] token}
[CLS] token đóng vai trò như "summary vector" của cả cặp câu:
\begin{itemize}
    \item Vị trí đầu tiên, có thể attend đến mọi token khác
    \item Qua các layer, tích lũy thông tin từ cả hai câu
    \item Biểu diễn cuối cùng chứa thông tin về mối quan hệ câu
    \item Được sử dụng cho mọi tác vụ classification sau này
\end{itemize}

\subsubsection{Combined Loss}
BERT được train với combined loss:
$$ \mathcal{L}_{total} = \mathcal{L}_{MLM} + \mathcal{L}_{NSP} $$

Cả hai loss được tối ưu đồng thời, giúp BERT học cả biểu diễn từ (MLM) và quan hệ câu (NSP).

\subsubsection{Tranh cãi về NSP}
Các nghiên cứu sau (RoBERTa, ALBERT) cho thấy:
\begin{itemize}
    \item NSP có thể không cần thiết hoặc thậm chí có hại cho một số tác vụ
    \item Vấn đề: Negative samples (random sentences) quá dễ phân biệt
    \item Cải tiến: Sentence Order Prediction (SOP) trong ALBERT - dự đoán thứ tự câu
\end{itemize}

Tuy nhiên, trong thiết kế gốc, NSP đóng góp vào thành công của BERT trên nhiều benchmark.

%% =========================================================================
%% CHƯƠNG 4: TINH CHỈNH (FINE-TUNING) BERT
%% =========================================================================
\section{Fine-tuning BERT cho các tác vụ Cụ thể}
\label{sec:fine_tuning_bert}
Sau khi được tiền huấn luyện trên khối lượng dữ liệu khổng lồ (BookCorpus + Wikipedia $\approx$ 3.3 tỷ từ), BERT có thể được fine-tune nhanh chóng cho các tác vụ cụ thể. Đây là minh chứng cho sức mạnh của transfer learning trong NLP.

\subsection{Quy trình fine-tuning Chung}
\label{ssec:quy_trinh_tinh_chinh}

\subsubsection{Nguyên lý cơ bản}
Fine-tuning BERT tuân theo triết lý "minimal architectural changes":
\begin{itemize}
    \item Giữ nguyên toàn bộ kiến trúc BERT đã pre-train
    \item Chỉ thêm một task-specific head (thường là 1-2 layers)
    \item Train toàn bộ parameters (BERT + head) trên dữ liệu của tác vụ
\end{itemize}

\subsubsection{So sánh với Feature-based Approach}
\begin{table}[H]
\centering
\caption{So sánh Fine-tuning vs Feature-based}
\begin{tabular}{lll}
\toprule
\textbf{Tiêu chí} & \textbf{Fine-tuning (BERT)} & \textbf{Feature-based (ELMo)} \\
\midrule
Parameters updated & Toàn bộ & Chỉ task-specific layers \\
Flexibility & Cao (adapt cho từng task) & Thấp (fixed features) \\
Training time & Lâu hơn & Nhanh hơn \\
Performance & Tốt nhất & Tốt \\
\bottomrule
\end{tabular}
\end{table}

\subsubsection{Lợi thế của Fine-tuning}
\begin{enumerate}
    \item \textbf{Task-specific adaptation:} Mô hình được điều chỉnh cho tác vụ cụ thể khác nhau
    \item \textbf{End-to-end optimization:} Gradient flow từ output đến input embeddings
    \item \textbf{Superior performance:} Thường đạt kết quả tốt hơn feature-based
\end{enumerate}

\subsection{Ví dụ fine-tuning cho các Tác vụ Phổ biến}
\label{ssec:vi_du_tinh_chinh}

\subsubsection{Sentence Pair Classification}
\textbf{Task:} Natural Language Inference, Paraphrase Detection, etc.

\textbf{Input format:}
\begin{verbatim}
[CLS] Premise text [SEP] Hypothesis text [SEP]
\end{verbatim}

\textbf{Architecture:}
\begin{itemize}
    \item Lấy final hidden state của [CLS]: $h_{[CLS]} \in \mathbb{R}^H$
    \item Classification layer: $W \in \mathbb{R}^{K \times H}$ (K = số classes)
    \item Output: $P(y|x) = \text{softmax}(h_{[CLS]} \cdot W^T)$
\end{itemize}

\textbf{Ví dụ - Natural Language Inference:}
\begin{itemize}
    \item Premise: "A man is playing guitar"
    \item Hypothesis: "A person is making music"
    \item Classes: Entailment, Contradiction, Neutral
    \item Expected: Entailment
\end{itemize}

\subsubsection{Single Sentence Classification}  
\textbf{Task:} Sentiment Analysis, Topic Classification, etc.

\textbf{Input format:}
\begin{verbatim}
[CLS] Single sentence text [SEP]
\end{verbatim}

\textbf{Architecture:} Tương tự sentence pair nhưng chỉ có một câu

\textbf{Ví dụ - Sentiment Analysis:}
\begin{itemize}
    \item Input: "This movie exceeded all my expectations!"
    \item Classes: Positive, Negative
    \item Expected: Positive
\end{itemize}

\subsubsection{Question Answering}
\textbf{Task:} Extract answer span from context (SQuAD style)

\textbf{Input format:}
\begin{verbatim}
[CLS] Question [SEP] Context paragraph [SEP]
\end{verbatim}

\textbf{Architecture đặc biệt:}
\begin{itemize}
    \item Học 2 vectors: Start vector $S$ và End vector $E$
    \item Với mỗi token $i$ trong context:
    \begin{itemize}
        \item $P_{start}(i) = \frac{e^{S \cdot T_i}}{\sum_j e^{S \cdot T_j}}$
        \item $P_{end}(i) = \frac{e^{E \cdot T_i}}{\sum_j e^{E \cdot T_j}}$
    \end{itemize}
    \item Answer = span từ $\arg\max_i P_{start}(i)$ đến $\arg\max_j P_{end}(j)$ với $j \geq i$
\end{itemize}

\textbf{Ví dụ:}
\begin{itemize}
    \item Question: "When was BERT published?"
    \item Context: "BERT was published by Google AI in October 2018..."
    \item Answer span: "October 2018"
\end{itemize}

\subsubsection{Token Classification (NER)}
\textbf{Task:} Gán nhãn cho từng token

\textbf{Architecture:}
\begin{itemize}
    \item Mỗi token output $T_i$ qua classification layer riêng
    \item $P(y_i|x) = \text{softmax}(T_i \cdot W^T)$
\end{itemize}

\textbf{Ví dụ - Named Entity Recognition:}
\begin{verbatim}
Input:  [CLS] John works at Google in London [SEP]
Labels:   O   B-PER  O    O  B-ORG  O  B-LOC   O
\end{verbatim}

\begin{figure}[H]
    \centering
    \includegraphics[width=\textwidth]{img/pretraining_finetuning.png}
    \caption{Fine-tuning BERT cho các tác vụ khác nhau. Lưu ý cách sử dụng [CLS] cho classification tasks và individual tokens cho sequence labeling. Architecture thay đổi tối thiểu - chủ yếu là output layer.}
    \label{fig:bert_finetuning_tasks}
\end{figure}

\subsubsection{Hyperparameters cho Fine-tuning}
Từ bài báo gốc, các hyperparameters được khuyến nghị:
\begin{itemize}
    \item \textbf{Batch size:} 16, 32
    \item \textbf{Learning rate:} 2e-5, 3e-5, 5e-5 
    \item \textbf{Epochs:} 2, 3, 4
    \item \textbf{Warmup:} 0.1 of total steps
    \item \textbf{Dropout:} 0.1 (giữ nguyên từ pre-training)
\end{itemize}

Lưu ý: Datasets nhỏ nhạy cảm hơn với hyperparameters. Nên thử nhiều random seeds.

%% =========================================================================
%% CHƯƠNG 5: THỰC NGHIỆM SENTIMENT ANALYSIS
%% =========================================================================
\section{Thực nghiệm với BERT: Phân loại Cảm xúc Văn bản}
\label{sec:thuc_nghiem_bert}
Để minh họa khả năng thực tế của BERT, chương này trình bày chi tiết quy trình áp dụng BERT cho bài toán phân loại cảm xúc (Sentiment Analysis) - một trong những ứng dụng phổ biến nhất của NLP.

\subsection{Lựa chọn Bài toán và Bộ dữ liệu}
\label{ssec:lua_chon_bai_toan_dataset}

\subsubsection{Bài toán Phân loại Cảm xúc}
Phân loại cảm xúc nhị phân là bài toán xác định một đoạn văn bản thể hiện thái độ tích cực hay tiêu cực. Đây là bài toán quan trọng với nhiều ứng dụng:
\begin{itemize}
    \item Phân tích phản hồi khách hàng
    \item Theo dõi dư luận trên mạng xã hội
    \item Đánh giá sản phẩm tự động
\end{itemize}

\subsubsection{Bộ dữ liệu IMDB Movie Reviews}
\textbf{Thông tin chung:}
\begin{itemize}
    \item \textbf{Kích thước:} 50,000 reviews (25,000 train + 25,000 test)
    \item \textbf{Cân bằng:} 50\% positive, 50\% negative
    \item \textbf{Độ dài:} Trung bình ~230 từ/review
    \item \textbf{Chất lượng:} Reviews được viết bởi người dùng thực với rating rõ ràng
\end{itemize}

\textbf{Ví dụ dữ liệu:}
\begin{itemize}
    \item \textbf{Positive:} "This film is a masterpiece of storytelling. The acting is superb, the cinematography breathtaking, and the plot keeps you engaged throughout..."
    \item \textbf{Negative:} "I was extremely disappointed with this movie. The plot was predictable, the dialogue felt forced, and the pacing was terribly slow..."
\end{itemize}

\subsection{Quy trình Thực nghiệm Chi tiết}
\label{ssec:quy_trinh_thuc_nghiem_sa}

\subsubsection{Tiền xử lý Dữ liệu}

\textbf{1. Làm sạch văn bản:}
\begin{verbatim}
import re
from bs4 import BeautifulSoup

def clean_text(text):
    # Loại bỏ HTML tags
    text = BeautifulSoup(text, "html.parser").get_text()
    # Loại bỏ URLs
    text = re.sub(r'http\S+', '', text)
    # Giữ lại dấu câu quan trọng
    text = re.sub(r'[^a-zA-Z0-9\s\.\,\!\?]', '', text)
    # Chuẩn hóa khoảng trắng
    text = ' '.join(text.split())
    return text
\end{verbatim}

\textbf{2. Tokenization với BERT Tokenizer:}
\begin{verbatim}
from transformers import BertTokenizer

tokenizer = BertTokenizer.from_pretrained('bert-base-uncased')

# Ví dụ tokenization
text = "This movie is fantastic!"
tokens = tokenizer.tokenize(text)
# Output: ['this', 'movie', 'is', 'fantastic', '!']

# Thêm special tokens và convert to IDs
encoded = tokenizer.encode_plus(
    text,
    add_special_tokens=True,     # Thêm [CLS] và [SEP]
    max_length=256,              # Padding/Truncation
    pad_to_max_length=True,
    return_attention_mask=True,   # Tạo attention mask
    return_tensors='pt'          # Return PyTorch tensors
)
\end{verbatim}

\textbf{3. Cấu trúc dữ liệu sau tokenization:}
\begin{itemize}
    \item \textbf{input\_ids:} [101, 2023, 3185, 2003, 10392, 999, 102, 0, 0, ...]
    \item \textbf{attention\_mask:} [1, 1, 1, 1, 1, 1, 1, 0, 0, ...]
    \item \textbf{token\_type\_ids:} [0, 0, 0, 0, 0, 0, 0, 0, 0, ...]
\end{itemize}

Trong đó: 101=[CLS], 102=[SEP], 0=[PAD]

\subsubsection{Xây dựng Mô hình}

\textbf{1. Load pre-trained BERT và thêm classification head:}
\begin{verbatim}
from transformers import BertForSequenceClassification

model = BertForSequenceClassification.from_pretrained(
    'bert-base-uncased',
    num_labels=2,  # Binary classification
    output_attentions=False,
    output_hidden_states=False
)
\end{verbatim}

\textbf{2. Kiến trúc classification head:}
Internally, BertForSequenceClassification thêm:
\begin{itemize}
    \item Dropout(0.1) layer
    \item Linear(768, 2) layer
    \item Loss function: CrossEntropyLoss
\end{itemize}

\subsubsection{Huấn luyện (Fine-tuning)}

\textbf{1. Thiết lập training parameters:}
\begin{verbatim}
from transformers import AdamW, get_linear_schedule_with_warmup

# Hyperparameters
EPOCHS = 3
BATCH_SIZE = 16
LEARNING_RATE = 2e-5
WARMUP_STEPS = 0.1 * total_steps
MAX_LENGTH = 256

# Optimizer với weight decay
optimizer = AdamW(
    model.parameters(),
    lr=LEARNING_RATE,
    eps=1e-8,
    weight_decay=0.01
)

# Learning rate scheduler
scheduler = get_linear_schedule_with_warmup(
    optimizer,
    num_warmup_steps=WARMUP_STEPS,
    num_training_steps=total_steps
)
\end{verbatim}

\textbf{2. Training loop với monitoring:}
\begin{verbatim}
from tqdm import tqdm
import torch

for epoch in range(EPOCHS):
    model.train()
    total_loss = 0
    
    for batch in tqdm(train_dataloader, desc=f"Epoch {epoch+1}"):
        # Move to GPU if available
        input_ids = batch['input_ids'].to(device)
        attention_mask = batch['attention_mask'].to(device)
        labels = batch['labels'].to(device)
        
        # Forward pass
        outputs = model(
            input_ids=input_ids,
            attention_mask=attention_mask,
            labels=labels
        )
        
        loss = outputs.loss
        total_loss += loss.item()
        
        # Backward pass
        loss.backward()
        
        # Gradient clipping
        torch.nn.utils.clip_grad_norm_(model.parameters(), 1.0)
        
        # Update weights
        optimizer.step()
        scheduler.step()
        optimizer.zero_grad()
    
    # Validation sau mỗi epoch
    avg_train_loss = total_loss / len(train_dataloader)
    val_accuracy = evaluate(model, val_dataloader)
    
    print(f"Epoch {epoch+1}:")
    print(f"  Training Loss: {avg_train_loss:.4f}")
    print(f"  Validation Accuracy: {val_accuracy:.4f}")
\end{verbatim}

\subsection{Kết quả Thực nghiệm}
\label{ssec:danh_gia_ket_qua_sa}

\subsubsection{Metrics đánh giá}
Chúng tôi sử dụng các metrics sau để đánh giá toàn diện:

\begin{table}[H]
    \centering
    \caption{Kết quả phân loại cảm xúc trên IMDB test set}
    \label{tab:imdb_results_actual}
    \begin{tabular}{lc}
        \toprule
        \textbf{Metric} & \textbf{Giá trị} \\
        \midrule
        Accuracy & 0.9426 \\
        Precision (Positive) & 0.9385 \\
        Recall (Positive) & 0.9472 \\
        F1-score (Positive) & 0.9428 \\
        Precision (Negative) & 0.9468 \\
        Recall (Negative) & 0.9380 \\
        F1-score (Negative) & 0.9424 \\
        \midrule
        Macro avg F1 & 0.9426 \\
        Weighted avg F1 & 0.9426 \\
        \bottomrule
    \end{tabular}
\end{table}

\subsubsection{Confusion Matrix}
\begin{verbatim}
                     Predicted
                  Positive  Negative
Actual  Positive    11,840    660
        Negative    775       11,725
\end{verbatim}

\subsubsection{So sánh với Baseline}
\begin{table}[H]
    \centering
    \caption{So sánh BERT với các phương pháp khác trên IMDB}
    \label{tab:comparison_methods}
    \begin{tabular}{lcc}
        \toprule
        \textbf{Phương pháp} & \textbf{Accuracy} & \textbf{Training Time} \\
        \midrule
        Logistic Regression + TF-IDF & 0.8847 & 5 phút \\
        Bi-LSTM & 0.8922 & 2 giờ \\
        CNN for Text & 0.8976 & 1.5 giờ \\
        BERT-base (our implementation) & \textbf{0.9426} & 3 giờ \\
        \midrule
        State-of-the-art (ULMFiT) & 0.9540 & 5 giờ \\
        \bottomrule
    \end{tabular}
\end{table}

\subsection{Phân tích và Thảo luận}
\label{ssec:thao_luan_huong_phat_trien_sa}

\subsubsection{Phân tích Kết quả}

\textbf{1. Hiệu suất vượt trội:}
BERT đạt accuracy 94.26\%, vượt xa các phương pháp truyền thống (~88-89\%). Điều này chứng tỏ sức mạnh của pre-trained representations.

\textbf{2. Phân tích lỗi:}
Từ confusion matrix, ta thấy:
\begin{itemize}
    \item False Negatives (775) > False Positives (660)
    \item BERT có xu hướng "lạc quan" hơn một chút
\end{itemize}

\textbf{3. Các trường hợp khó:}
Phân tích manual các mẫu bị phân loại sai cho thấy:
\begin{itemize}
    \item \textbf{Sarcasm/Irony:} "What a 'masterpiece'! I've never wasted 2 hours so wonderfully."
    \item \textbf{Mixed sentiment:} "Great acting but terrible plot"
    \item \textbf{Domain-specific:} Reviews với nhiều thuật ngữ điện ảnh chuyên ngành
\end{itemize}

\subsubsection{Visualization: Attention Analysis}
Một điểm thú vị là có thể visualize attention weights để hiểu BERT "chú ý" vào đâu:

\begin{verbatim}
# Ví dụ: "This movie is absolutely terrible"
# Attention weights từ [CLS] token:
# "terrible": 0.42 (highest)
# "absolutely": 0.18
# "movie": 0.12
# "is": 0.08
# "This": 0.06
\end{verbatim}

BERT học được rằng "terrible" là từ quan trọng nhất cho sentiment.

\subsubsection{Thách thức Gặp phải}

\textbf{1. Tài nguyên tính toán:}
\begin{itemize}
    \item GPU memory: Cần ít nhất 8GB VRAM cho batch size 16
    \item Training time: ~1 giờ/epoch trên Tesla T4
    \item Solution: Gradient accumulation, mixed precision training
\end{itemize}

\textbf{2. Hyperparameter tuning:}
\begin{itemize}
    \item Learning rate rất nhạy cảm (2e-5 vs 3e-5 có thể chênh 2\% accuracy)
    \item Cần thử nhiều random seeds
    \item Solution: Bayesian optimization hoặc grid search
\end{itemize}

\textbf{3. Overfitting trên small datasets:}
\begin{itemize}
    \item BERT có 110M parameters nhưng chỉ fine-tune trên 25K mẫu
    \item Risk: Catastrophic forgetting
    \item Solution: Early stopping, regularization, data augmentation
\end{itemize}

\subsubsection{Hướng Cải thiện}

\textbf{1. Advanced models:}
\begin{itemize}
    \item RoBERTa: Training lâu hơn, dynamic masking → +0.5-1\% accuracy
    \item ALBERT: Parameter sharing → Nhẹ hơn 70\% với performance tương đương
    \item DistilBERT: Knowledge distillation → Nhanh hơn 60\%, chỉ giảm 3\% accuracy
\end{itemize}

\textbf{2. Domain adaptation:}
\begin{itemize}
    \item Continue pre-training BERT trên movie reviews corpus
    \item Tạo Movie-BERT chuyên biệt cho domain
\end{itemize}

\textbf{3. Ensemble methods:}
\begin{itemize}
    \item Kết hợp predictions từ nhiều fine-tuned models
    \item Hoặc ensemble BERT với CNN/LSTM
\end{itemize}

\textbf{4. Data augmentation:}
\begin{itemize}
    \item Back-translation: English → French → English
    \item Paraphrasing với T5 hoặc GPT
    \item Adversarial examples generation
\end{itemize}

%% =========================================================================
%% CHƯƠNG 6: KẾT QUẢ VÀ SO SÁNH
%% =========================================================================
\section{Kết quả và So sánh}
\label{sec:ket_qua_so_sanh}
Phần này trình bày kết quả của BERT trên các benchmark chuẩn từ bài báo gốc, cung cấp bối cảnh để hiểu tầm quan trọng của BERT trong lịch sử NLP.

\subsection{Kết quả trên GLUE Benchmark}
\label{ssec:ket_qua_glue}
GLUE (General Language Understanding Evaluation) là bộ benchmark gồm 9 tác vụ đa dạng, được thiết kế để đánh giá khả năng hiểu ngôn ngữ tổng quát của mô hình.

\subsubsection{Mô tả các tác vụ GLUE}
\begin{itemize}
    \item \textbf{MNLI:} Multi-Genre Natural Language Inference (393K mẫu)
    \item \textbf{QQP:} Quora Question Pairs - Xác định câu hỏi trùng lặp (364K mẫu)
    \item \textbf{QNLI:} Question NLI - Đoạn văn có chứa câu trả lời? (105K mẫu)
    \item \textbf{SST-2:} Stanford Sentiment Treebank - Phân loại cảm xúc (67K mẫu)
    \item \textbf{CoLA:} Corpus of Linguistic Acceptability (8.5K mẫu)
    \item \textbf{STS-B:} Semantic Textual Similarity (7K mẫu)
    \item \textbf{MRPC:} Microsoft Research Paraphrase Corpus (3.7K mẫu)
    \item \textbf{RTE:} Recognizing Textual Entailment (2.5K mẫu)
\end{itemize}

\begin{table}[H]
    \centering
    \caption{Kết quả BERT trên GLUE test set. Điểm số cao nhất được in đậm.}
    \label{tab:glue_results_detailed}
    \resizebox{\textwidth}{!}{%
    \begin{tabular}{lcccccccccc}
        \toprule
        \textbf{Model} & \textbf{MNLI-m/mm} & \textbf{QQP} & \textbf{QNLI} & \textbf{SST-2} & \textbf{CoLA} & \textbf{STS-B} & \textbf{MRPC} & \textbf{RTE} & \textbf{WNLI} & \textbf{Avg} \\
        & (Acc) & (F1/Acc) & (Acc) & (Acc) & (Mcc) & (Corr) & (F1/Acc) & (Acc) & (Acc) & \\
        \midrule
        Previous SOTA & 80.6/80.1 & 66.1/- & 82.3 & 93.2 & 35.0 & 81.0 & 86.0/- & 61.7 & 65.1 & 74.0 \\
        OpenAI GPT & 82.1/81.4 & 70.3/88.5 & 87.4 & 91.3 & 45.4 & 80.0 & 82.3/75.7 & 56.0 & 65.1 & 72.8 \\
        \midrule
        BERT\textsubscript{BASE} & 84.6/83.4 & 71.2/89.2 & 90.5 & 93.5 & 52.1 & 85.8 & 88.9/84.8 & 66.4 & 65.1 & 78.3 \\
        BERT\textsubscript{LARGE} & \textbf{86.7/85.9} & \textbf{72.1/89.3} & \textbf{92.7} & \textbf{94.9} & \textbf{60.5} & \textbf{86.5} & \textbf{89.3/85.4} & \textbf{70.1} & 65.1 & \textbf{80.5} \\
        \bottomrule
    \end{tabular}%
    }
\end{table}

\subsubsection{Phân tích kết quả GLUE}
\begin{itemize}
    \item BERT\textsubscript{LARGE} đạt 80.5 điểm trung bình, tăng 7.7 điểm tuyệt đối so với GPT
    \item Cải thiện lớn nhất: CoLA (+15.1), RTE (+14.1) - các tác vụ có ít dữ liệu
    \item Điều này cho thấy BERT học được representations tổng quát, transfer tốt cho small datasets
\end{itemize}

\subsection{Kết quả trên SQuAD}
\label{ssec:ket_qua_squad}

\subsubsection{SQuAD v1.1}
SQuAD v1.1 yêu cầu trích xuất câu trả lời từ đoạn văn cho câu hỏi.

\begin{table}[H]
    \centering
    \caption{Kết quả trên SQuAD v1.1 test set}
    \label{tab:squad_v1_results}
    \begin{tabular}{lcc}
        \toprule
        \textbf{Model} & \textbf{EM} & \textbf{F1} \\
        \midrule
        Human Performance & 82.3 & 91.2 \\
        Previous best (ensemble) & 84.4 & 90.9 \\
        \midrule
        BERT\textsubscript{BASE} (single) & 80.8 & 88.5 \\
        BERT\textsubscript{LARGE} (single) & 84.1 & 90.9 \\
        BERT\textsubscript{LARGE} (ensemble) & \textbf{87.4} & \textbf{93.2} \\
        \bottomrule
    \end{tabular}
\end{table}

Điểm đáng chú ý: BERT ensemble vượt qua human performance!

\subsubsection{SQuAD v2.0}
SQuAD v2.0 khó hơn vì có thêm câu hỏi không có câu trả lời trong đoạn văn.

\begin{table}[H]
    \centering
    \caption{Kết quả trên SQuAD v2.0 test set}
    \label{tab:squad_v2_results}
    \begin{tabular}{lcc}
        \toprule
        \textbf{Model} & \textbf{EM} & \textbf{F1} \\
        \midrule
        Human Performance & 86.8 & 89.5 \\
        Previous best & 72.3 & 74.8 \\
        \midrule
        BERT\textsubscript{LARGE} & \textbf{78.7} & \textbf{81.9} \\
        \bottomrule
    \end{tabular}
\end{table}

BERT cải thiện 7.1 F1 points - một bước nhảy lớn cho tác vụ khó này.

\subsection{Ablation Studies - Hiểu sâu về BERT}
\label{ssec:ablation_study}

\subsubsection{Tầm quan trọng của các thành phần}
\begin{table}[H]
    \centering
    \caption{Ablation study: Loại bỏ từng thành phần của BERT}
    \label{tab:ablation_components}
    \begin{tabular}{lccccc}
        \toprule
        \textbf{Model} & \textbf{MNLI-m} & \textbf{QNLI} & \textbf{MRPC} & \textbf{SST-2} & \textbf{SQuAD} \\
        & (Acc) & (Acc) & (Acc) & (Acc) & (F1) \\
        \midrule
        BERT\textsubscript{BASE} & 84.4 & 88.4 & 86.7 & 92.7 & 88.5 \\
        \midrule
        No NSP & 83.9 & 84.9 & 86.5 & 92.6 & 87.9 \\
        No MLM (LTR only) & 82.1 & 84.3 & 77.5 & 92.1 & 77.8 \\
        \bottomrule
    \end{tabular}
\end{table}

Insights:
\begin{itemize}
    \item Bỏ NSP: Giảm nhẹ performance, đặc biệt trên QNLI (-3.5\%)
    \item Bỏ MLM (chỉ train left-to-right): Giảm mạnh, đặc biệt SQuAD (-10.7 F1)
    \item → Bidirectional training (MLM) quan trọng hơn NSP
\end{itemize}

\subsubsection{Effect của Model Size}
\begin{table}[H]
    \centering
    \caption{Ảnh hưởng của kích thước mô hình}
    \label{tab:model_size_effect}
    \begin{tabular}{lcccccc}
        \toprule
        \textbf{Model} & \textbf{\#Layers} & \textbf{\#Heads} & \textbf{Hidden} & \textbf{Params} & \textbf{MNLI} & \textbf{MRPC} \\
        \midrule
        BERT-Small & 4 & 8 & 512 & 29M & 77.6 & 79.8 \\
        BERT-Medium & 8 & 8 & 512 & 42M & 80.8 & 82.7 \\
        BERT-Base & 12 & 12 & 768 & 110M & 84.4 & 86.7 \\
        BERT-Large & 24 & 16 & 1024 & 340M & 86.6 & 87.8 \\
        \bottomrule
    \end{tabular}
\end{table}

Larger models consistently better, even on small datasets (MRPC: 3.7K samples).

%% =========================================================================
%% CHƯƠNG 7: HẠN CHẾ VÀ HƯỚNG PHÁT TRIỂN
%% =========================================================================
\section{Hạn chế và Hướng Phát triển}
\label{sec:han_che_huong_phat_trien}

\subsection{Hạn chế của BERT}
\label{ssec:han_che_bert}
Mặc dù BERT đã tạo ra bước đột phá, mô hình vẫn có những hạn chế cần được nhận diện và giải quyết.

\subsubsection{Chi phí tính toán khổng lồ}
\begin{itemize}
    \item \textbf{Pre-training cost:}
    \begin{itemize}
        \item BERT\textsubscript{BASE}: 4 ngày trên 16 TPU chips
        \item BERT\textsubscript{LARGE}: 4 ngày trên 64 TPU chips
        \item Chi phí ước tính: \$7,000 - \$50,000 tùy cấu hình
    \end{itemize}
    \item \textbf{Environmental impact:} Carbon footprint tương đương 5 chuyến bay xuyên Mỹ
    \item \textbf{Inference cost:} BERT\textsubscript{LARGE} cần ~1GB memory, chậm cho real-time apps
\end{itemize}

\subsubsection{Masked Token Discrepancy}
\begin{itemize}
    \item [MASK] token chỉ xuất hiện trong pre-training, không có trong fine-tuning/inference
    \item Mặc dù có chiến lược 80-10-10, vẫn tồn tại domain shift
    \item Các mô hình sau (ELECTRA) giải quyết bằng cách không dùng [MASK]
\end{itemize}

\subsubsection{Fixed Length Limitation}
\begin{itemize}
    \item Maximum 512 tokens - không đủ cho văn bản dài (legal documents, books)
    \item Truncation làm mất thông tin quan trọng
    \item Solutions: Longformer (4096 tokens), BigBird (sparse attention)
\end{itemize}

\subsubsection{Unidirectional Fine-tuning for Generation}
\begin{itemize}
    \item BERT là encoder-only, không tự nhiên cho text generation
    \item Các hack (mask prediction iteratively) kém hiệu quả
    \item Dẫn đến phát triển encoder-decoder models (T5, BART)
\end{itemize}

\subsubsection{Static Embeddings}
\begin{itemize}
    \item Position embeddings cố định 512 vị trí
    \item Không thể extrapolate cho sequences dài hơn
    \item Relative position embeddings (T5) linh hoạt hơn
\end{itemize}

\subsubsection{NSP Task Controversy}
\begin{itemize}
    \item Nhiều nghiên cứu cho thấy NSP không cần thiết hoặc có hại
    \item Random sentences quá dễ phân biệt → task không học gì có ích
    \item RoBERTa bỏ NSP và đạt kết quả tốt hơn
\end{itemize}

\subsection{Các Hướng Phát triển Kế thừa BERT}
\label{ssec:huong_phat_trien_ke_thua}
BERT đã mở ra "Cambrian explosion" của các mô hình NLP. Các hướng phát triển chính:

\subsubsection{Cải tiến Training Efficiency}

\textbf{1. RoBERTa (Robustly Optimized BERT):}
\begin{itemize}
    \item Loại bỏ NSP, train lâu hơn (500K steps vs 100K)
    \item Dynamic masking thay vì static
    \item Larger batches (8K vs 256)
    \item Kết quả: +2-3\% trên hầu hết tasks
\end{itemize}

\textbf{2. ELECTRA (Efficiently Learning an Encoder):}
\begin{itemize}
    \item Replaced token detection thay vì MLM
    \item Train discriminator phân biệt real/fake tokens
    \item Hiệu quả hơn 4x về compute
    \item Small ELECTRA $\approx$ BERT\textsubscript{BASE} với 1/4 compute
\end{itemize}

\subsubsection{Model Compression}

\textbf{1. DistilBERT:}
\begin{itemize}
    \item Knowledge distillation từ BERT\textsubscript{BASE}
    \item 40\% smaller, 60\% faster, giữ 97\% performance
    \item Key: Distill during pre-training, not just fine-tuning
\end{itemize}

\textbf{2. ALBERT (A Lite BERT):}
\begin{itemize}
    \item Factorized embeddings: V × E → V × E' × H (E' << H)
    \item Cross-layer parameter sharing
    \item 18x fewer parameters với comparable performance
\end{itemize}

\textbf{3. Quantization \& Pruning:}
\begin{itemize}
    \item 8-bit quantization: 4x compression, <1\% performance drop
    \item Structured pruning: Remove attention heads/layers
    \item TensorRT optimization cho deployment
\end{itemize}

\subsubsection{Architectural Innovations}

\textbf{1. Handling Long Documents:}
\begin{itemize}
    \item \textbf{Longformer:} Sliding window + global attention
    \item \textbf{BigBird:} Sparse attention (random + window + global)
    \item \textbf{Linformer:} Linear complexity attention
\end{itemize}

\textbf{2. Unified Models:}
\begin{itemize}
    \item \textbf{T5:} "Text-to-Text" framework - mọi task là seq2seq
    \item \textbf{BART:} Denoising autoencoder cho generation
    \item \textbf{UniLM:} Unified pre-training cho cả understanding và generation
\end{itemize}

\subsubsection{Multilingual và Cross-lingual}

\textbf{1. mBERT → XLM → XLM-R:}
\begin{itemize}
    \item Progressive improvements trong multilingual understanding
    \item XLM-R: 100 languages, 2.5TB text
    \item Zero-shot cross-lingual transfer
\end{itemize}

\textbf{2. Language-Specific Models:}
\begin{itemize}
    \item PhoBERT (Vietnamese), CamemBERT (French), BERTje (Dutch)
    \item Often outperform mBERT on monolingual tasks
\end{itemize}

\subsubsection{Domain Adaptation}

\textbf{1. Continued Pre-training:}
\begin{itemize}
    \item BioBERT: PubMed papers → biomedical NER/RE
    \item SciBERT: Scientific papers → citation intent classification
    \item FinBERT: Financial texts → sentiment analysis
\end{itemize}

\textbf{2. Adapter Modules:}
\begin{itemize}
    \item Thêm small trainable modules, freeze BERT
    \item Efficient multi-domain adaptation
    \item Parameter efficient: ~3\% additional parameters per task
\end{itemize}

\subsubsection{Toward Foundation Models}
BERT's success paved the way for:
\begin{itemize}
    \item GPT-3: Scaling to 175B parameters
    \item PaLM: 540B parameters với breakthrough capabilities
    \item ChatGPT/GPT-4: Instruction following và alignment
\end{itemize}

%% =========================================================================
%% KẾT LUẬN
%% =========================================================================
\section{Kết luận}
\label{sec:ket_luan}
BERT - Bidirectional Encoder Representations from Transformers - không chỉ là một mô hình mà còn là một cột mốc quan trọng đánh dấu sự chuyển mình của lĩnh vực Xử lý Ngôn ngữ Tự nhiên. Bằng cách giới thiệu phương pháp học biểu diễn hai chiều sâu sắc thông qua Masked Language Modeling và Next Sentence Prediction, BERT đã chứng minh rằng việc pre-training trên dữ liệu không nhãn có thể tạo ra những biểu diễn ngôn ngữ mạnh mẽ, có khả năng transfer xuất sắc cho nhiều tác vụ downstream.

\textbf{Những đóng góp then chốt của BERT:}
\begin{itemize}
    \item \textbf{Deeply bidirectional architecture:} Lần đầu tiên cho phép mỗi từ được hiểu trong ngữ cảnh đầy đủ của cả câu, không chỉ một phía như các mô hình trước đó.
    
    \item \textbf{Pre-training objectives sáng tạo:} MLM và NSP tuy đơn giản nhưng hiệu quả, cho phép học từ dữ liệu không nhãn - nguồn tài nguyên dồi dào và dễ thu thập.
    
    \item \textbf{Transfer learning paradigm:} Thiết lập mô hình "pre-train once, fine-tune for everything" đã trở thành chuẩn mực cho ngành.
    
    \item \textbf{Empirical breakthroughs:} Không chỉ cải thiện incremental mà tạo ra những bước nhảy lớn trên nhiều benchmarks, thậm chí vượt qua human performance.
\end{itemize}

\textbf{Tác động sâu rộng:}

BERT đã mở ra kỷ nguyên mới của "Foundation Models" - những mô hình nền tảng được pre-train trên quy mô lớn rồi adapted cho các ứng dụng cụ thể. Từ BERT, chúng ta đã chứng kiến sự phát triển bùng nổ của:
\begin{itemize}
    \item Các biến thể hiệu quả hơn (RoBERTa, ELECTRA, ALBERT)
    \item Mô hình đa ngôn ngữ (XLM-R, mT5)
    \item Scaling lên quy mô khổng lồ (GPT-3, PaLM, ChatGPT)
    \item Ứng dụng thực tế trong mọi ngành (search, translation, chatbots, content generation)
\end{itemize}

\textbf{Bài học cho tương lai:}

Thành công của BERT dạy chúng ta rằng:
\begin{enumerate}
    \item Simple ideas executed well can be revolutionary (MLM là ý tưởng đơn giản nhưng hiệu quả)
    \item Scale matters - cả về dữ liệu và model capacity
    \item Transfer learning là chìa khóa cho AI dân chủ hóa
    \item Bidirectionality và context là cốt lõi của language understanding
\end{enumerate}

\textbf{Nhìn về phía trước:}

Trong khi BERT đã đặt nền móng vững chắc, vẫn còn nhiều thách thức:
\begin{itemize}
    \item Computational efficiency cho deployment rộng rãi
    \item Handling longer contexts và multimodal inputs
    \item Better alignment với human values và intentions
    \item Multilingual và low-resource language support
\end{itemize}

BERT sẽ được ghi nhớ không chỉ vì những con số ấn tượng trên benchmarks, mà vì đã thay đổi cách chúng ta nghĩ về language understanding và machine learning. Từ một bài báo học thuật, BERT đã trở thành nền tảng cho vô số ứng dụng AI đang thay đổi thế giới hàng ngày. Đó chính là di sản thực sự của một nghiên cứu đột phá.